\subchapter{Lab3: Add a custom application}{Add a new recipe to support a required
        custom application}

During this lab, you will:
\begin{itemize}
  \item Write a recipe for a custom application
  \item Integrate this application into the build
\end{itemize}

This is the first step of adding an application to Yocto. The
remaining part is covered in the next lab, "Create a Yocto layer".

\section{Setup and organization}

In this lab we will add a recipe handling the \code{nInvaders} application
found at \url{https://ninvaders.sourceforge.net/}. Before starting the recipe
itself, find the \code{recipes-extended} directory originating from
OpenEmbedded-Core and add a subdirectory for your application.

\section{First hands on nInvaders}

The nInvaders application is a terminal based game following the space invaders
family. In order to deal with the text based user interface, nInvaders uses the
ncurses library.

First try to find the project homepage, download the sources and have a first
look: license, Makefile, requirements\dots

\section{Write a minimal recipe}

Create a file that respects the Yocto nomenclature: \code{${PN}_${PV}.bb}
%stopzone

Specify the source URL of the latest nInvaders archive and give a try at
building your recipe:

\begin{verbatim}
bitbake ninvaders
\end{verbatim}

\section{Archive checksum and license}

BitBake will refuse to go any further if it cannot validate the downloaded
bundle using a checksum. You'll also need to provide some information about the
license of the package.

\section{Testing and troubleshooting}

Try to make the recipe on your own. Also eliminate the warnings related to your
recipe: some configuration variables are not mandatory but it is a very good
practice to define them all.

If you hang on a problem, check the following points:
\begin{itemize}
  \item The checksum and the URI are valid
  \item The dependencies are explicitly defined
  \item The internal state has changed, clean the working directory: \\
     \code{bitbake -c cleanall ninvaders}
\end{itemize}

One of the build failures you will face will generate many messages such as
\code{multiple definition of `skill_level'; aliens.o:(.bss+0x674):
  first defined here}.

The \code{multiple definition} issue is due to the code base of {\em
nInvaders} being quite old, and having multiple compilation units
redefine the same symbols. While this was accepted by older {\em gcc}
versions, since {\em gcc 10} this is no longer accepted by default.

While we could fix the {\em nInvaders} code base, we will take a
different route: ask {\em gcc} to behave as it did before {\em gcc 10}
and accept such redefinitions. This can be done by passing the
\code{-fcommon} {\em gcc} flag.

To achieve this, make sure to add \code{-fcommon} to the \yoctovar{CFLAGS}
variable.

Tip: BitBake has command line flags to increase its verbosity and activate debug
outputs. Also, remember that you need to cross-compile nInvaders for ARM! Maybe,
you will have to configure your recipe to resolve some mistakes done in the
application's Makefile (which is often the case). A bitbake variable permits
to add some Makefile's options, you should look for it.

\section{Update the rootfs and test}

Now that you've compiled the \code{nInvaders} application, generate a new
rootfs image with \code{bitbake core-image-minimal}. Then update the
NFS root directory. You can confirm the \code{nInvaders} program is
present by running:
\begin{verbatim}
find /nfs -iname ninvaders
\end{verbatim}

Access the board command line through SSH. You should be able to
launch the \code{nInvaders} program. Now, it's time to play!

\section{Inspect the build}

The \code{nInvaders} application was unpacked and compiled in the
recipe's work directory. Can you spot \code{nInvaders}' directory in
the build work directory?

Once you found it, look around. You should at least spot some
directories:
\begin{itemize}
    \item The sources. Remember the \yoctovar{S} variable?
    \item \code{temp}. There are two kinds of files in there. Can you
      tell what are their purposes?
    \item Try to see if the licences of \code{nInvaders} were
      extracted.
\end{itemize}
