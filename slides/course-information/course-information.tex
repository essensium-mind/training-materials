\begin{frame}

\frametitle{Participate!}
During the lectures...
\begin{itemize}
\item Don't hesitate to ask questions. Other people in the audience may have
similar questions too.
\item Don't hesitate to share your experience too, for example to compare Linux
with other operating systems you know.
\item Your point of view is most valuable, because it can be similar to your
colleagues' and different from the trainer's.
\item In on-line sessions
   \begin{itemize}
   \item Please always keep your camera on!
   \item Also make sure your name is properly filled.
   \item You can also use the "Raise your hand" button when you wish to
         ask a question but don't want to interrupt.
   \end{itemize}
\item All this helps the trainer to engage with participants, see when
something needs clarifying and make the session more interactive, enjoyable
and useful for everyone.
\end{itemize}
\end{frame}

\begin{frame}
\frametitle{Collaborate!}
\begin{columns}
\column{0.8\textwidth}
  As in the Free Software and Open Source community,
  collaboration between participants is valuable in this training session:
  \begin{itemize}
    \item Use the dedicated Matrix channel for this session
	  to add questions.
    \item If your session offers practical labs, you can also
          report issues, share screenshots and command output there.
    \item Don't hesitate to share your own answers and to help others
          especially when the trainer is unavailable.
    \item The Matrix channel is also a good place to ask questions
	  outside of training hours, and after the course is over.
    \end{itemize}
\column{0.2\textwidth}
  \includegraphics[height=0.8\textheight]{slides/course-information/matrix-screenshot.png}
\end{columns}
\end{frame}
