\section{Optimizing applications}

\begin{frame}
\frametitle{Measuring: strace}
\begin{itemize}
        \item Allows to trace all the system calls made by an
              application and its children.
        \item Useful to:
        \begin{itemize}
                \item Understand how time is spent in user space
                \item For example, easy to find file open attempts (\code{open()}),
                      file access (\code{read()}, \code{write()}), and
                      memory allocations (\code{mmap2()}). Can be done
                      without any access to source code!
                \item Find the biggest time consumers
                      (low hanging fruit)
                \item Find unnecessary work done in applications
                      and scripts. Example: opening the same file(s)
                      multiple times, or trying to open files that
                      do not exist.
        \end{itemize}
        \item Limitation: you can't trace the \code{init} process!
\end{itemize}
\end{frame}

\input{../common/strace.tex}
\input{../common/ltrace.tex}
\begin{frame}{Valgrind}
  \begin{columns}[T]
    \column{0.8\textwidth}
    \url{https://valgrind.org/}
    \begin{itemize}
    \item {\em instrumentation framework for building dynamic analysis tools}
      \begin{itemize}
      \item detect many memory management and threading bugs
      \item profile programs
      \end{itemize}
    \item Supported architectures: x86, x86-64, ARMv7, ARMv8, mips32,
      s390, ppc32 and ppc64
    \item Very popular tool especially for debugging memory issues
    \item Runs your program on a synthetic CPU $\rightarrow$
      significant performance impact (100 x slower on SAMA5D3!),
      but very detailed instrumentation
    \item Runs on the target. Easy to build with Yocto Project
          or Buildroot.
    \end{itemize}
    \column{0.2\textwidth}
    \includegraphics[width=\textwidth]{common/valgrind1.png}
  \end{columns}
\end{frame}

\begin{frame}{Valgrind tools}
  \begin{itemize}
  \item {\em Memcheck}: detects memory-management problems
  \item {\em Cachegrind}: cache profiler, detailed simulation of the
    I1, D1 and L2 caches in your CPU and so can accurately pinpoint
    the sources of cache misses in your code
  \item {\em Callgrind}: extension to Cachegrind, provides extra
    information about call graphs
  \item {\em Massif}: performs detailed heap profiling by taking
    regular snapshots of a program's heap
  \item {\em Helgrind}: thread debugger which finds data races in
    multithreaded programs. Looks for memory locations accessed by
    multiple threads without locking.
  \item More at \url{https://valgrind.org/info/tools.html}
  \end{itemize}
\end{frame}

\begin{frame}[fragile]{Valgrind examples}
  \begin{itemize}
  \item {\em Memcheck}
    \begin{block}{}
      {\tiny
\begin{verbatim}
$ valgrind --leak-check=yes <program>
  ==19182== Invalid write of size 4
  ==19182==    at 0x804838F: f (example.c:6)
  ==19182==    by 0x80483AB: main (example.c:11)
  ==19182==  Address 0x1BA45050 is 0 bytes after a block of size 40 alloc'd
  ==19182==    at 0x1B8FF5CD: malloc (vg_replace_malloc.c:130)
  ==19182==    by 0x8048385: f (example.c:5)
  ==19182==    by 0x80483AB: main (example.c:11)
\end{verbatim}
      }
    \end{block}

  \item {\em Callgrind}
    \begin{block}{}
      {\tiny
\begin{verbatim}
$ valgrind --tool=callgrind --dump-instr=yes --simulate-cache=yes --collect-jumps=yes <program>
$ ls callgrind.out.*
callgrind.out.1234
$ callgrind_annotate callgrind.out.1234
\end{verbatim}
      }
    \end{block}
  \end{itemize}
\end{frame}

\begin{frame}{Kcachegrind - Visualizing Valgrind profiling data}
  \begin{center}
    \includegraphics[height=0.8\textheight]{common/kcachegrind.png}
    \url{https://github.com/KDE/kcachegrind}
  \end{center}
\end{frame}


\begin{frame}[fragile]
\frametitle{perf}
\begin{itemize}
        \item Uses hardware performance counters, much faster than Valgrind!
        \item Need a kernel with \kconfig{CONFIG_PERF_EVENTS} and \kconfig{CONFIG_HW_PERF_EVENTS}
        \item User space tool: \code{perf}. It is part of the kernel
                sources so it is always in sync with your kernel.
        \item Usage:
        \begin{block}{}
\begin{verbatim}
perf record /my/command
\end{verbatim}
        \end{block}
        \item Get the results with:
        \begin{block}{}
\begin{verbatim}
perf report
\end{verbatim}
        \end{block}
        \item Note: advice to run \code{perf} on a filesystem built with
              glibc. Didn't manage to compile \code{perf} on a Musl
              root filesystem (Buildroot 2021.02 status). Once again, glibc is
              recommended for debugging.
\end{itemize}
\end{frame}

\begin{frame}[fragile]
\frametitle{perf report output}
\begin{block}{}
\tiny
\begin{verbatim}
# To display the perf.data header info, please use --header/--header-only options.
#
#
# Total Lost Samples: 0
#
# Samples: 5K of event 'cycles'
# Event count (approx.): 1392529663
#
# Overhead  Command  Shared Object             Symbol
# ........  .......  ........................  ....................................
#
    10.72%  ffmpeg   [kernel.kallsyms]         [k] video_get_user
    10.60%  ffmpeg   [kernel.kallsyms]         [k] vector_swi
     4.76%  ffmpeg   libc-2.31.so              [.] ioctl
     4.22%  ffmpeg   [kernel.kallsyms]         [k] __se_sys_ioctl
     3.81%  ffmpeg   [kernel.kallsyms]         [k] __video_do_ioctl
     3.42%  ffmpeg   libavformat.so.58.45.100  [.] avformat_find_stream_info
     2.83%  ffmpeg   [kernel.kallsyms]         [k] video_usercopy
     2.70%  ffmpeg   libc-2.31.so              [.] cfree
     2.58%  ffmpeg   [kernel.kallsyms]         [k] __fget_light
     2.53%  ffmpeg   libpthread-2.31.so        [.] __errno_location
     2.40%  ffmpeg   [kernel.kallsyms]         [k] arm_copy_from_user
     2.26%  ffmpeg   [kernel.kallsyms]         [k] memset
     2.09%  ffmpeg   [kernel.kallsyms]         [k] mutex_unlock
     2.06%  ffmpeg   [kernel.kallsyms]         [k] v4l2_ioctl
     2.05%  ffmpeg   libavcodec.so.58.91.100   [.] av_init_packet
     1.95%  ffmpeg   libc-2.31.so              [.] memset
...
\end{verbatim}
\end{block}
\end{frame}

\setuplabframe
{Optimizing the application}
{
\begin{itemize}
\item Compile the video player with just the features needed at run
      time.
\item Trace and profile the video player with \code{strace}
\item Observe size and time savings
\end{itemize}
}

