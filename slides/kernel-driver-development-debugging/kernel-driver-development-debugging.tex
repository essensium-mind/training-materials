\section{Kernel debugging}

\input{../common/printk.tex}

\begin{frame}
  \frametitle{DebugFS}
  A virtual filesystem to export debugging information to user space.
  \begin{itemize}
  \item Kernel configuration: \kconfig{CONFIG_DEBUG_FS}
    \begin{itemize}
    \item \code{Kernel hacking -> Debug Filesystem}
    \end{itemize}
  \item The debugging interface disappears when Debugfs is
    configured out.
  \item You can mount it as follows:
    \begin{itemize}
    \item \code{sudo mount -t debugfs none /sys/kernel/debug}
    \end{itemize}
  \item First described on \url{https://lwn.net/Articles/115405/}
  \item API documented in the Linux Kernel Filesystem API:
    \kdochtml{filesystems/debugfs}
    {The debugfs filesystem}
  \end{itemize}
\end{frame}

\begin{frame}[fragile]
  \frametitle{DebugFS API}
  \begin{itemize}
  \item Create a sub-directory for your driver:
    \begin{itemize}
    \item \mint{c}+struct dentry *debugfs_create_dir(const char *name,+
      \mint{c}+struct dentry *parent);+
    \end{itemize}
  \item Expose an integer as a file in DebugFS. Example:
    \begin{itemize}
    \item \mint{c}+struct dentry *debugfs_create_u8+
      \mint{c}+(const char *name, mode_t mode, struct dentry *parent,+
      \mint{c}+u8 *value);+
      \begin{itemize}
      \item \code{u8}, \code{u16}, \code{u32}, \code{u64} for decimal representation
      \item \code{x8}, \code{x16}, \code{x32}, \code{x64} for hexadecimal representation
      \end{itemize}
    \end{itemize}
  \item Expose a binary blob as a file in DebugFS:
    \begin{itemize}
    \item \mint{c}+struct dentry *debugfs_create_blob(const char *name,+
      \mint{c}+mode_t mode, struct dentry *parent,+
      \mint{c}+struct debugfs_blob_wrapper *blob);+
    \end{itemize}
  \item Also possible to support writable DebugFS files or customize
    the output using the more generic \kfunc{debugfs_create_file}
    function.
  \end{itemize}
\end{frame}

\begin{frame}
  \frametitle{Deprecated debugging mechanisms}
  Some additional debugging mechanisms, whose usage is now
  considered deprecated
  \begin{itemize}
  \item Adding special \code{ioctl()} commands for debugging
    purposes. DebugFS is preferred.
  \item Adding special entries in the \code{proc} filesystem. DebugFS is
    preferred.
  \item Adding special entries in the \code{sysfs} filesystem. DebugFS is
    preferred.
  \item Using \kfunc{printk}. The \code{pr_*()} and \code{dev_*()}
    functions are preferred.
  \end{itemize}
\end{frame}

\begin{frame}[fragile]
  \frametitle{Using Magic SysRq}
  Functionnality provided by serial drivers
  \begin{itemize}
  \item Allows to run multiple debug / rescue commands even when the
    kernel seems to be in deep trouble
    \begin{itemize}
    \item On PC: press \code{[Alt]} + \code{[Prnt Scrn]} + \code{<character>}
          simultaneously\\
          (\code{[SysRq]} = \code{[Alt]} + \code{[Prnt Scrn]})
    \item On embedded: in the console, send a break character\\
      (Picocom: press \code{[Ctrl]} + \code{a} followed by \code{[Ctrl]}
      + \code{\ }), then press \code{<character>}
    \end{itemize}
  \item Example commands:
    \begin{itemize}
    \item \code{h}: show available commands
    \item \code{s}: sync all mounted filesystems
    \item \code{b}: reboot the system
    \item \code{n}: makes RT processes nice-able.
    \item \code{w}: shows the kernel stack of all sleeping processes
    \item \code{t}: shows the kernel stack of all running processes
    \item You can even register your own!
    \end{itemize}
  \item Detailed in \kdochtml{admin-guide/sysrq}
  \end{itemize}
\end{frame}

\input{../common/kgdb.tex}

\begin{frame}
  \frametitle{Debugging with a JTAG interface}
  Two types of JTAG dongles
  \begin{itemize}
  \item The ones offering a \code{gdb} compatible interface, over a
    serial port or an Ethernet connection. \code{gdb} can directly
    connect to them.
  \item The ones not offering a gdb compatible interface are generally
    supported by OpenOCD (Open On Chip Debugger):
    \url{http://openocd.sourceforge.net/}
    \begin{itemize}
    \item OpenOCD is the bridge between the gdb debugging language
      and the JTAG interface of the target CPU.
    \item See the very complete documentation:
      \url{https://openocd.org/pages/documentation.html}
    \item For each board, you'll need an OpenOCD configuration file
      (ask your supplier)
    \end{itemize}
  \end{itemize}
   \begin{center}
     \includegraphics[width=\textwidth]{slides/kernel-driver-development-debugging/jtag.pdf}
   \end{center}
\end{frame}

\begin{frame}
  \frametitle{Early traces}
  \begin{itemize}
  \item If something breaks before the \code{tty} layer, serial driver
    and serial console are properly registered, you might just have
    nothing else after "\code{Starting kernel...}"
  \item On ARM, if your platform implements it, you can activate
    (\kconfig{CONFIG_DEBUG_LL} and \kconfig{CONFIG_EARLYPRINTK}), and add
    \code{earlyprintk} to the kernel command line
    \begin{itemize}
    \item Assembly routines to just push a character and wait for it to
      be sent
    \item Extremely basic, but is part of the uncompressed section, so
      available even if the kernel does not uncompress correctly!
    \end{itemize}
  \item On other platforms, hoping that your serial driver implements
    \kfunc{OF_EARLYCON_DECLARE}, you can enable \kconfig{SERIAL_EARLYCON}
    \begin{itemize}
    \item The kernel will try to hook an appropriate \code{earlycon}
      UART driver using the \code{stdout-path} of the device-tree.
    \end{itemize}
  \end{itemize}
\end{frame}

\begin{frame}
  \frametitle{More kernel debugging tips}
  \begin{itemize}
  \item Make sure \kconfig{CONFIG_KALLSYMS_ALL} is enabled
    \begin{itemize}
    \item To get oops messages with symbol names instead of raw
      addresses
    \item Turned on by default
    \end{itemize}
  \item Make sure \kconfig{CONFIG_DEBUG_INFO} is also enabled
    \begin{itemize}
    \item This way, the kernel is compiled with \code{$(CROSSCOMPILE)gcc
        -g}, which keeps the source code inside the binaries.
    \end{itemize}
  \item If your device is not probed, try enabling
        \kconfig{CONFIG_DEBUG_DRIVER}
    \begin{itemize}
    \item Extremely verbose!
    \item Will enable all the debug logs in the device-driver core
      section
    \end{itemize}
  \end{itemize}
\end{frame}

\begin{frame}
  \frametitle{Getting help and reporting bugs}
  \begin{itemize}
  \item If you are using a custom kernel from a hardware vendor, contact
        that company. The community will have less interest supporting
        a custom kernel.
  \item Otherwise, or if this doesn't work, try to reproduce the
        issue on the latest version of the kernel.
  \item Make sure you investigate the issue as much as you can: see
    \kdochtml{admin-guide/bug-bisect}
  \item Check for previous bugs reports. Use web search engines,
    accessing public mailing list archives.
  \item If you're the first to face the issue, it's very useful for
    others to report it, even if you cannot investigate it further.
  \item If the subsystem you report a bug on has a mailing list, use
    it. Otherwise, contact the official maintainer (see the
    \kfile{MAINTAINERS} file). Always give as many useful details as
    possible.
  \end{itemize}
\end{frame}

\setuplabframe
{Kernel debugging}
{
  \begin{itemize}
  \item Use the dynamic debug feature.
  \item Add debugfs entries
  \item Load a broken driver and see it crash
  \item Analyze the error information dumped by the kernel.
  \item Disassemble the code and locate the exact C instruction which
    caused the failure.
  \end{itemize}
}
